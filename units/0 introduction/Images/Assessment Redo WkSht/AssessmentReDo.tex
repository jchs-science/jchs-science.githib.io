\documentclass[14pt, fleqn, paper=letter, oneside]{scrartcl}

\newcommand{\includehead}{true}
\newcommand{\includefoot}{true}

% set basic page format
\usepackage[headsepline=\includehead, footsepline=\includefoot]{scrlayer-scrpage} 
\usepackage[margin=0.375in,
    footskip=1.5\baselineskip,
    headsep=0.5\baselineskip,
    includehead=\includehead, includefoot=\includefoot]{geometry}   %Fixed margins
\usepackage[compact]{titlesec}

%\usepackage{setspace}
%\onehalfspace
%\doublespace    

% image support
\renewcommand{\topfraction}{0.85}   %Fixes float spacing
\renewcommand{\textfraction}{0.1}
\renewcommand{\floatpagefraction}{0.75}
\usepackage{graphicx}
	\graphicspath{{/Users/fred/Library/TeXShop/Images/}{./Images/}}
\usepackage[space]{grffile}		

% symbol support
\usepackage{siunitx}    %SI unit : \si{\'unit'} or \SI{#}{\'unit'}
\sisetup{detect-all}
\usepackage{chemfig}    %Write chemical formulas
\usepackage{mathtools}  %Basic math an extension of amsmath

% formatting support
\usepackage{multicol}   %Multiple cols body : command \multicols{#}{'text'}
\usepackage{multirow}   %Multiple row spanning in tables : \multirow{#}{width}{'text'}
\usepackage{enumitem}   %Enumeration control
\usepackage{hyperref} % hyperlinks
\usepackage{soul} % highlighting with \hl{command}


% font and date format
\usepackage{newtxtext}
\setkomafont{disposition}{\bfseries}
\usepackage{fancyref}           %Automatically adds Table (tab:' ') or Figure (fig:' ')
\usepackage{texdate}
\initcurrdate
\def\setdateformat{e\ b\ y\ }
\renewcommand{\headfont}{\normalfont}
\renewcommand{\footfont}{\normalfont}

% useful commands
\newcommand{\biu}[1]{\textbf{\emph{\underline{#1}}}}
\newcommand{\centerframe}[1]{ % this command makes a box around the content
    {\centering\fbox{\begin{minipage}{0.975\columnwidth}#1\end{minipage}}}}

% document commands
%\ihead{Name:}
%\chead{Hour:}
%\ohead{Date:}
\ifoot{Revised \printdate}
\cfoot{\maintitle}
\ofoot{\thepage}
\newcommand{\maintitle}{Assessment Redo - Worksheet}

%===================================
\begin{document}
\section*{\maintitle}
In order to re-submit your introduction assessment for re-grading, you must also submit this worksheet and get at least 90\% on it.
You will probably have to use outside sources beyond the course material to help you complete this worksheet.


\begin{minipage}[t]{0.5\textwidth}
\section*{Units}
\renewcommand{\arraystretch}{1.5}
{\center
\begin{tabular}{|l|p{25mm}|c|c|}
\hline
	&\textbf{Unit}
	&\textbf{Symbol}
	&\begin{tabular}{c}\textbf{\small Equivalent}\\\textbf{\small Combination}\end{tabular}
\\\hline
	\small{Mass} &&&
\\\hline
	\small{Length} &&&
\\\hline
	\small{Time} &&&
\\\hline
	\small{Current} &&&
\\\hline
	\small{Temperature} &&&
\\\hline
	\small{Force} &&&
\\\hline
	\small{Velocity} &&&
\\\hline
	\small{Acceleration} &&&
\\\hline
	 \small{Volume-liq} &&&
\\\hline
 	 \small{Volume-sol} &&&
\\\hline
	\small{Area} &&&
\\\hline
	\small{Power} &&&
\\\hline
	\small{Density} &&&
\\\hline
	\small{Energy} &&&
\\\hline
	\small{Frequency} &&&
\\\hline
	\small{Current} &&&
\\\hline
	\small{Angle} &&&
\\\hline
	\small{Pressure} &&&
\\\hline
\end{tabular}

}
\end{minipage}
\hfill
\begin{minipage}[t]{0.4\textwidth}
\section*{Conversions}
\renewcommand{\arraystretch}{1.5}
\begin{tabular}{r @{ $\Rightarrow$ \rule{20mm}{0.4pt} }c}
\SI{100}{mm} & \si{m}
\\
\SI{1}{A} & \si{\micro A}
\\
\SI{0.89}{Gs} & \si{Ms}
\\
\SI{0.02}{N} & \si{mN}
\\
\SI{3141}{W} & \si{kw}
\\
\SI{1234567}{\hertz} & \si{\mega\hertz}
\\
\SI{0.000387}{L} & \si{\micro L}
\\
\SI{3}{GJ} & \si{MJ}
\\
\SI{1867}{g} & \si{kg}
\\
\SI{0.045}{V} & \si{mV}
\\
\SI{3274}{nL} & \si{mL}
\\
\SI{1234}{mg} & \si{g}
\\
\SI{0.6745}{kW} & \si{W}
\\
\SI{86,400}{s} & \si{ks}
\\
\SI{123}{m\ohm} & \si{\ohm}
\\
\SI{0.456}{mF} & \si{\micro F}
\\
\SI{273}{MW} & \si{GW}
\\
\SI{200000}{m} & \si{km}
\\
\SI{31415926}{mm} & \si{km}
\\
\SI{62831852}{\micro g} & \si{g}
\\
\SI{2022}{kJ} & \si{MJ}
\end{tabular}
\end{minipage}


\clearpage
\section*{Communication}
Make an observation, come up with a model to explain your observation, and come up with a test for your model.  Here is an example;

\begin{quote}
Question:
Why do sidewalks have cracks at regular intervals?
Model: Maybe each section is the same amount of concrete that one wheel-barrow will hold.
Test: I'll look up the volume of a wheel barrow and estimate the volume of a section of side walk and see if they are the same.
\end{quote}

\newcommand{\writtenlinespace}{3mm}
\noindent\rule{\textwidth}{0.4pt}\\[\writtenlinespace]
\noindent\rule{\textwidth}{0.4pt}\\[\writtenlinespace]
\noindent\rule{\textwidth}{0.4pt}\\[\writtenlinespace]
\noindent\rule{\textwidth}{0.4pt}\\[\writtenlinespace]
\noindent\rule{\textwidth}{0.4pt}\\[\writtenlinespace]
\noindent\rule{\textwidth}{0.4pt}\\[\writtenlinespace]
\noindent\rule{\textwidth}{0.4pt}\\[\writtenlinespace]
\noindent\rule{\textwidth}{0.4pt}\\[\writtenlinespace]
\noindent\rule{\textwidth}{0.4pt}\\[\writtenlinespace]
\noindent\rule{\textwidth}{0.4pt}\\[\writtenlinespace]
\noindent\rule{\textwidth}{0.4pt}\\[\writtenlinespace]
\noindent\rule{\textwidth}{0.4pt}\\[\writtenlinespace]
\noindent\rule{\textwidth}{0.4pt}\\[\writtenlinespace]
\noindent\rule{\textwidth}{0.4pt}\\[\writtenlinespace]
\noindent\rule{\textwidth}{0.4pt}\\[\writtenlinespace]
\noindent\rule{\textwidth}{0.4pt}\\[\writtenlinespace]
\noindent\rule{\textwidth}{0.4pt}\\[\writtenlinespace]
\noindent\rule{\textwidth}{0.4pt}\\[\writtenlinespace]
\noindent\rule{\textwidth}{0.4pt}\\[\writtenlinespace]
\noindent\rule{\textwidth}{0.4pt}\\[\writtenlinespace]









































\end{document}