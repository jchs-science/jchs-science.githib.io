\documentclass[13pt, fleqn, paper=letter, oneside]{scrartcl}

\newcommand{\includehead}{false}
\newcommand{\includefoot}{true}

% set basic page format
\usepackage[headsepline=\includehead, footsepline=\includefoot]{scrlayer-scrpage} 
\usepackage[margin=0.375in,
    footskip=1.5\baselineskip,
    headsep=0.5\baselineskip,
    includehead=\includehead, includefoot=\includefoot]{geometry}   %Fixed margins
\usepackage[compact]{titlesec}

%\usepackage{setspace}
%\onehalfspace
%\doublespace    

% image support
\renewcommand{\topfraction}{0.85}   %Fixes float spacing
\renewcommand{\textfraction}{0.1}
\renewcommand{\floatpagefraction}{0.75}
\usepackage{graphicx}
	\graphicspath{{/Users/fred/Library/TeXShop/Images/}{./Images/}}
\usepackage[space]{grffile}		

% symbol support
\usepackage{siunitx}    %SI unit : \si{\'unit'} or \SI{#}{\'unit'}
\sisetup{detect-all}
\usepackage{chemfig}    %Write chemical formulas
\usepackage{mathtools}  %Basic math an extension of amsmath

% formatting support
\usepackage{multicol}   %Multiple cols body : command \multicols{#}{'text'}
\usepackage{multirow}   %Multiple row spanning in tables : \multirow{#}{width}{'text'}
\usepackage{enumitem}   %Enumeration control
\usepackage{hyperref} % hyperlinks
\usepackage{soul} % highlighting with \hl{command}
\usepackage{array}


% font and date format
\usepackage{newtxtext}
\setkomafont{disposition}{\bfseries}
\usepackage{fancyref}           %Automatically adds Table (tab:' ') or Figure (fig:' ')
\usepackage{texdate}
\initcurrdate
\def\setdateformat{e\ b\ y\ }
\renewcommand{\headfont}{\normalfont}
\renewcommand{\footfont}{\normalfont}

% useful commands
\newcommand{\biu}[1]{\textbf{\emph{\underline{#1}}}}
\newcommand{\centerframe}[1]{ % this command makes a box around the content
    {\centering\fbox{\begin{minipage}{0.975\columnwidth}#1\end{minipage}}}}

% document commands
%\ihead{Name:}
%\chead{Hour:}
%\ohead{Date:}
\ifoot{Revised \printdate}
\cfoot{\maintitle}
\ofoot{\thepage}
\newcommand{\maintitle}{Introduction Assessment - Due 11/12 Sep}

%===================================
\begin{document}
\section{\maintitle}
You have been asked to build a structure that will hold the most weight. 
However, due to global warming, the only materials we have left to build with is spaghetti and mini-marshmallows!
Your group will have to work together to design the best structure and then create a detailed design.

\section{Assignment}
Your structure must holds \emph{at least} 1kg, 75mm above the table, using only:
\begin{itemize}
\item 75g of spaghetti
\item 20 mini-marshmallows
\end{itemize}

You will NOT be graded on if your structure succeeds or not.
You will be graded on how well you communicate about your structure.
This means that everyone turns in an individual 'proposal' to be considered.

You only need to have two sections, a drawing that has enough labels to reproduce your design, and an analysis of whether or not your design passed with improvements you would do and WHY you would make those improvements.

\section{Proficiency Scales}
\begin{table}[h]
\centering
\newcommand{\colwidth}{37mm}
\begin{tabular}{|c | >{\raggedright\arraybackslash}p{\colwidth}
									 | >{\raggedright\arraybackslash}p{\colwidth}
									 | >{\raggedright\arraybackslash}p{\colwidth}
									 | >{\raggedright\arraybackslash}p{\colwidth}
									 |}
\hline
	&\textbf{4 - Surpasses}
	&\textbf{3 - Meets}
	&\textbf{2 - Approaching}
	&\textbf{1 - Below}
\\\hline
\begin{tabular}{c}
\textbf{Metric}\\\textbf{System}
\end{tabular}
 & Student can accurately measure and apply the correct units and convert within the metric system.
 & Student can accurately measure and apply the correct units.  
 & Student can measure and apply the correct units with minimal errors. 
 & Student cannot measure or apply the correct units.
\\\hline
\textbf{Communication}
 & Student can provide a description and a labeled diagram with measurement of their project.

\vspace{3mm}

 Student can describe why the structure failed and tested solution.
 & Student can provide a complete description or a labeled diagram with measurement of their project.

\vspace{3mm}

	Student can describe why the structure failed and a written solution. 
 
 & Student can provide a description or an unlabeled diagram with some measurements of their project.

\vspace{3mm}

	Student can describe why the structure failed.

 & Student can provide a generic description with no diagram. 

\vspace{3mm}

Student can only identify if the structure was a success or failure 



\\\hline
\end{tabular}
\end{table}

\newpage
\section*{Proposal by: \rule{80mm}{1pt} - Hr: \rule{20mm}{1pt}}
\section*{Proposed design (detailed drawing with measurements)}

\vfill
\newcommand{\linespacing}{3mm}
\section*{Analysis of Structure Test}
\mbox{}
\\[\linespacing]\rule{\textwidth}{0.4pt}
\\[\linespacing]\rule{\textwidth}{0.4pt}
\\[\linespacing]\rule{\textwidth}{0.4pt}
\\[\linespacing]\rule{\textwidth}{0.4pt}
\\[\linespacing]\rule{\textwidth}{0.4pt}
\\[\linespacing]\rule{\textwidth}{0.4pt}
\\[\linespacing]\rule{\textwidth}{0.4pt}
\\[\linespacing]\rule{\textwidth}{0.4pt}
\\[\linespacing]\rule{\textwidth}{0.4pt}
\\[\linespacing]\rule{\textwidth}{0.4pt}
\\[\linespacing]\rule{\textwidth}{0.4pt}
\\[\linespacing]\rule{\textwidth}{0.4pt}
\\[\linespacing]\rule{\textwidth}{0.4pt}
\\[\linespacing]\rule{\textwidth}{0.4pt}












































\end{document}