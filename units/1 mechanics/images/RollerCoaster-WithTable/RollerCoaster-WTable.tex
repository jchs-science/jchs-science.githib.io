\documentclass[12pt, fleqn, paper=letter]{scrartcl}

\newcommand{\includehead}{true}
\newcommand{\includefoot}{true}

% set basic page format
\usepackage[headsepline=\includehead, footsepline=\includefoot]{scrlayer-scrpage} 
\usepackage[margin=0.5in,
    footskip=1.5\baselineskip,
    headsep=0.5\baselineskip,
    includehead=\includehead, includefoot=\includefoot]{geometry}   %Fixed margins
\usepackage[compact]{titlesec}

%\usepackage{setspace}
%\onehalfspace
%\doublespace    

% image support
\renewcommand{\topfraction}{0.85}   %Fixes float spacing
\renewcommand{\textfraction}{0.1}
\renewcommand{\floatpagefraction}{0.75}
\usepackage{graphicx}
	\graphicspath{{/Users/fred/Library/TeXShop/Images/}{./Images/}}
\usepackage[space]{grffile}		

% symbol support
\usepackage{siunitx}    %SI unit : \si{\'unit'} or \SI{#}{\'unit'}
\sisetup{detect-all}
\usepackage{chemfig}    %Write chemical formulas
\usepackage{mathtools}  %Basic math an extension of amsmath

% formatting support
\usepackage{multicol}   %Multiple cols body : command \multicols{#}{'text'}
\usepackage{multirow}   %Multiple row spanning in tables : \multirow{#}{width}{'text'}
\usepackage{enumitem}   %Enumeration control
\usepackage{hyperref} % hyperlinks
\usepackage{soul} % highlighting with \hl{command}
\usepackage{tabularx}



% font and date format
\usepackage{newtxtext}
\setkomafont{disposition}{\bfseries}
\usepackage{fancyref}           %Automatically adds Table (tab:' ') or Figure (fig:' ')
\usepackage{texdate}
\initcurrdate
\def\setdateformat{e\ b\ y\ }
\renewcommand{\headfont}{\normalfont}
\renewcommand{\footfont}{\normalfont}

% useful commands
\newcommand{\biu}[1]{\textbf{\emph{\underline{#1}}}}
\newcommand{\centerframe}[1]{ % this command makes a box around the content
    {\centering\fbox{\begin{minipage}{0.975\columnwidth}#1\end{minipage}}}}

% document commands
%\ihead{\hfill Name:\hspace{2in} Hour:\hspace{1in}}
\chead{Name:}
\ohead{Hour:\hspace{1in}}
\ifoot{Revised \printdate}
\cfoot{\maintitle}
\ofoot{\thepage}
\newcommand{\maintitle}{Roller Coaster Assessment - with Calculation Table}

\usepackage{tikz}
\usepackage{xcolor}

%Document
%===================================
\begin{document}
\begin{center}
\huge \textbf{\maintitle}
\end{center}

We are losing customers at our amusement park!
Our advertising department has tasked your group to create a new roller coaster that will attract the masses.
This roller coaster must be fast, flashy, and, most importantly, safe!
One roller coaster idea will be selected from each class to be built at our amusement park.
At the end of your testing you will need the final specifications of your roller coaster to present to the advertising department.

Criteria:
\begin{itemize}
\item Your roller coaster must have at least two of the following features
	\begin{itemize}
	\item loop, hill, jump, spiral, corkscrew
	\end{itemize}

\item The marble must make it from the beginning hill to the cup at the end

\item The group must use the entire length of the tube for the roller coaster
	\begin{itemize}
	\item Hint, two pieces put together = total length in meters, measure this before you start!
	\end{itemize}

\item The roller coaster must be built in such a way that it can be taken apart at the end of class without damaging the foam tubing
	\begin{itemize}
	\item \textbf{\hl{ABSOLUTELY} no cutting, tearing, or bending the tubing!}
	\end{itemize}
\end{itemize}

\section*{Proficiency Scales - Student can...}
\newcommand{\tablecolwidth}{55mm}
{\centering
\begin{tabular}{|c|p{\tablecolwidth}|p{\tablecolwidth}|p{\tablecolwidth}|}
\hline
  &\textbf{Newton's 2nd Law}
  &\textbf{Momentum}
  &\textbf{Energy}
\\\hline
4
% Force
	&Calculate the force for multiple marble materials.

	\vspace{2mm}
	Compare and contrast the effect multiple marble materials has on the \emph{forces} in the roller coaster.
% Momentum
	&Calculate momentum for multiple marble materials. 
	
	\vspace{2mm}
	Compare and contrast the effect multiple marble materials has on the \emph{momentum} in the roller coaster.

	\vspace{2mm}
	Design a safe roller coaster, the cup that catches the marble moves less than 5cm.
% Energy
	&Calculate the potential and kinetic energy for multiple marble materials.
	
	\vspace{2mm}
	Compare and contrast the effect multiple marble materials has on the \emph{energy} in the roller coaster.
	
	\vspace{2mm}
	Calculate in a design feature how the energy changes and why.
\\\hline
3% Force
	&Calculate the force at some point in the roller coaster.

	\vspace{2mm}
	Define and describe the relationship between force, mass, and acceleration.
% Momentum
	&Calculate momentum at some point in the roller coaster. 
	
	\vspace{2mm}
	Design a safe roller coaster, the cup that catches the marble moves less than 10cm.
% Energy
	&Calculate the potential and kinetic energy at some point(s) in the roller coaster.
	
	\vspace{2mm}
	Explain in a design feature how the energy changes and why.
\\\hline
2% Force
	&Calculate the force at some point in the roller coaster, with minimal errors.

	\vspace{2mm}
	Define the relationship between force, mass, and acceleration.
% Momentum
	&Calculate momentum at some point in the roller coaster, with minimal errors. 
	
	\vspace{2mm}
	Design a safe roller coaster, the cup that catches the marble moves less than 15cm.
% Energy
	&Calculate the potential and kinetic energy at some point(s) in the roller coaster, with minimal errors.
	
	\vspace{2mm}
	Explain in a design feature how the energy changes.
\\\hline
1% Force
	&Build a roller coaster structure but can not use force to explain how it works.  
% Momentum
	&Build a roller coaster structure but can not use momentum to explain how it works.
% Energy
	&Build a roller coaster structure but can not use kinetic or potential to explain how it works.
\\\hline
\end{tabular}

}


\clearpage
\section*{Roller Coaster Design}
Draw a DETAILED picture of your roller coaster that includes measurements in the space below.
Be sure to reference heights (from the ground up), distances between track components, etc.
Someone else should be able to construct your track solely from this drawing.
\hl{Label the drawing with the letters listed in the table below to indicate where you measured those values.}

\vspace{5mm}

{\centering
	\begin{tikzpicture}[scale=.6]
    \foreach \x in {0,...,32}
    \foreach \y in {0,...,32}
    {
  \fill (\x,\y) circle (0.03cm);
    }       
  \end{tikzpicture}

}

\clearpage
\section*{Calculation Table}
The highlighted cells are the data that you need to have from your roller coaster.

\begin{table}[h]
\centering
\renewcommand{\arraystretch}{3.5}
\begin{tabularx}{\textwidth}{|c|X||X|X|X|}
\hline
\textbf{Quantity}&\hspace{10mm}\textbf{Equation}\hspace{10mm}
	&\textbf{Steel}&\textbf{Glass}&\textbf{Wood}
\\\hline
Mass && 0.0085 kg & 0.0037 kg & 0.0007 kg
\\\hline
\hl{Distance}&&&&\\\hline
\hl{Time}&&&&\\\hline
Velocity&&&&\\\hline
Acceleration&&&&\\\hline
Force&&&&\\\hline
Momentum&&&&\\\hline
Kinetic Energy&&&&\\\hline
\hl{Height}&&&&\\\hline
Potential Energy&&&&\\\hline
\end{tabularx}
\end{table}

\clearpage
\section*{Writeup}
Use the lines below to write anything else you need to get the grade you want.
When in doubt, explain WHY.
Why does your marble speed up as it goes down the hill, that sort of thing.
You should have at least 3 paragraphs; 1 for force, 1 for momentum, and 1 for energy.

\vspace{3mm}
\noindent\rule{\columnwidth}{0.4pt}
\\[3mm]\rule{\columnwidth}{0.4pt}
\\[3mm]\rule{\columnwidth}{0.4pt}
\\[3mm]\rule{\columnwidth}{0.4pt}
\\[3mm]\rule{\columnwidth}{0.4pt}
\\[3mm]\rule{\columnwidth}{0.4pt}
\\[3mm]\rule{\columnwidth}{0.4pt}
\\[3mm]\rule{\columnwidth}{0.4pt}
\\[3mm]\rule{\columnwidth}{0.4pt}
\\[3mm]\rule{\columnwidth}{0.4pt}
\\[3mm]\rule{\columnwidth}{0.4pt}
\\[3mm]\rule{\columnwidth}{0.4pt}
\\[3mm]\rule{\columnwidth}{0.4pt}
\\[3mm]\rule{\columnwidth}{0.4pt}
\\[3mm]\rule{\columnwidth}{0.4pt}
\\[3mm]\rule{\columnwidth}{0.4pt}
\\[3mm]\rule{\columnwidth}{0.4pt}
\\[3mm]\rule{\columnwidth}{0.4pt}
\\[3mm]\rule{\columnwidth}{0.4pt}
\\[3mm]\rule{\columnwidth}{0.4pt}
\\[3mm]\rule{\columnwidth}{0.4pt}
\\[3mm]\rule{\columnwidth}{0.4pt}
\\[3mm]\rule{\columnwidth}{0.4pt}
\\[3mm]\rule{\columnwidth}{0.4pt}
\\[3mm]\rule{\columnwidth}{0.4pt}
\\[3mm]\rule{\columnwidth}{0.4pt}



















\end{document}