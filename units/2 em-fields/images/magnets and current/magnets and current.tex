\documentclass[14pt, fleqn, paper=letter, oneside]{scrartcl}

\newcommand{\includehead}{true}
\newcommand{\includefoot}{true}

% set basic page format
\usepackage[headsepline=\includehead, footsepline=\includefoot]{scrlayer-scrpage} 
\usepackage[margin=0.5in,
    footskip=1.5\baselineskip,
    headsep=0.5\baselineskip,
    includehead=\includehead, includefoot=\includefoot]{geometry}   %Fixed margins
%\usepackage[compact]{titlesec}

%\usepackage{setspace}
%\onehalfspace
%\doublespace    

% image support
\renewcommand{\topfraction}{0.85}   %Fixes float spacing
\renewcommand{\textfraction}{0.1}
\renewcommand{\floatpagefraction}{0.75}
\usepackage{graphicx}
	\graphicspath{{/Users/fred/Library/TeXShop/Images/}{./Images/}}
\usepackage[space]{grffile}		

% symbol support
\usepackage{siunitx}    %SI unit : \si{\'unit'} or \SI{#}{\'unit'}
\sisetup{detect-all}
\usepackage{chemfig}    %Write chemical formulas
\usepackage{mathtools}  %Basic math an extension of amsmath

% formatting support
\usepackage{multicol}   %Multiple cols body : command \multicols{#}{'text'}
\usepackage{multirow}   %Multiple row spanning in tables : \multirow{#}{width}{'text'}
\usepackage{enumitem}   %Enumeration control
%\usepackage{hyperref} % hyperlink
\PassOptionsToPackage{hyphens}{url}\usepackage{hyperref}
\usepackage{soul} % highlighting with \hl{command}


% font and date format
\usepackage{newtxtext}
\setkomafont{disposition}{\bfseries}
\usepackage{fancyref}           %Automatically adds Table (tab:' ') or Figure (fig:' ')
\usepackage{texdate}
\initcurrdate
\def\setdateformat{e\ b\ y\ }
\renewcommand{\headfont}{\normalfont}
\renewcommand{\footfont}{\normalfont}

% useful commands
\newcommand{\biu}[1]{\textbf{\emph{\underline{#1}}}}
\newcommand{\centerframe}[1]{ % this command makes a box around the content
    {\centering\fbox{\begin{minipage}{0.975\columnwidth}#1\end{minipage}}}}

% document commands
\ihead{PLEASE DO NOT WRITE ON}
\chead{Classroom Copy}
\ohead{PLEASE DO NOT WRITE ON}
\ifoot{Revised \printdate}
\cfoot{\maintitle}
\ofoot{\thepage}
\newcommand{\maintitle}{Magnets and Current}

%===================================
\begin{document}
\section{Magnets}
You have learned about fields and charges, now it is time to learn about magnets and magnetism.

With the materials in front of you:v
various magnets, various materials, compass, magnetic film,
work through the following 4 prompts.

\hl{\textbf{Record all observations in your notebook!!!}}

When you have completed all three prompts, develop a model for how magnets work and write it up in your lab notebook.

\begin{multicols}{2}

\subsection{Playing with Magnets}
Use the magnets provided, and any of the materials around you to explore how magnets interact with each other and with the different materials.
\hl{In your lab notebook you should record:}
\begin{itemize}
\item How the magnets interact with each other - you should have different observations based on they type of magnet.

\item How the magnets interact with the different materials - you should have different observations based on the magnet and material.
\end{itemize}

\subsection{Investigate Magnets with a Compass}
Use the compass to investigate the magnets.
\hl{In your lab notebook you should cover the following:}
\begin{itemize}
\item Describe how the compass interacts with the different types of magnets
\item Explain how you think the compass works
\item Describe how the compass changes based on where it is relative to the magnet (for the different magnet types)
\end{itemize}

\subsection{Magnetic Film}
Use the magnetic film to find at least 5 magnets.
For each magnet you you find,
\hl{in your lab notebook record the following:}
\begin{itemize}
\item Where you found the magnet
\item Describe what you think the purpose of the magnet is
\end{itemize}

\subsection{Model}
Now that you have had some opportunities to explore magnets, come up with a model for how they work.
(Hint, think of what we talked about last time!)
\hl{In your lab notebook, record your model for a magnet, including any details you discovered during your first activities.}
\end{multicols}

\clearpage
\section{Magnets and Current}
We are going to zoom out a bit.
We use compasses to navigate on Earth because the Earth has a magnetic field.
That magnetic field does much, much, more for us than help with navigation.

A bit of background information.
The sun constantly sends out electric charges in the form of charged particles (more on those next semester).
Moving charges are called a current.
Turns out, magnets and currents interact.
After you read the materials provided, come up with what you think happens when magnets and currents interact.

\begin{multicols}{2}
\subsection{Earth's Magnetosphere}
Magnetosphere is a fancy word for magnetic field.
Since it is the magnetic field for a whole planet, it is special.

Read the article provided, titled \emph{Does the Earth's Magnetosphere Protect Us From the Sun's Solar Wind?}, or online at:

\url{https://sciencing.com/earths-magnetosphere-protects-suns-solar-wind-1955.html}

\hl{In your lab notebook, answer the following questions:}
\begin{itemize}
\item What is the solar wind made up of?
\item How does the magnetosphere protect life on earth?
\item How can solar winds effect electronics?
\item How does the magnetosphere protect Earth's atmosphere?
\end{itemize}

\subsection{Northern Lights}
Read the excerpt of the article provided, titled \emph{Northern Lights}, or online at:
\url{https://www.northernlightscentre.ca/northernlights.html}

\hl{In your lab notebook, answer the following questions:}
\begin{itemize}
\item What causes the northern lights?
\item Would the northern lights be brighter with or without the magnetosphere?
\end{itemize}


\subsection{Heliosphere}
Last but not least, go big or go home.

Read the excerpt of the article provided, titled \emph{Heliosphere}, or online at:
\url{https://science.nasa.gov/heliophysics/focus-areas/heliosphere}
\hl{In your lab notebook, answer the following questions:}
\begin{itemize}
\item What is the heliosphere made up of?
\item What does the heliosphere do for the earth?
\item Compare and contrast what you know about the heliosphere of our sun, and the magnetosphere of Earth.
\end{itemize}

\end{multicols}






































\end{document}