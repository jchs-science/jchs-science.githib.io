\documentclass[14pt, fleqn, paper=letter, oneside]{scrartcl}

\newcommand{\includehead}{true}
\newcommand{\includefoot}{true}

% set basic page format
\usepackage[headsepline=\includehead, footsepline=\includefoot]{scrlayer-scrpage} 
\usepackage[margin=0.5in,
%    footskip=1.5\baselineskip, headsep=0.5\baselineskip,
    includehead=\includehead, includefoot=\includefoot]{geometry}   %Fixed margins
\usepackage[compact]{titlesec}

%\usepackage{setspace}
%\onehalfspace
%\doublespace    

% image support
\renewcommand{\topfraction}{0.95}   %Fixes float spacing
\renewcommand{\textfraction}{0.01}
\renewcommand{\floatpagefraction}{0.85}
\usepackage{graphicx}
	\graphicspath{{/Users/fred/Library/TeXShop/Images/}{./Images/}}
\usepackage[space]{grffile}		

% symbol support
\usepackage{siunitx}    %SI unit : \si{\'unit'} or \SI{#}{\'unit'}
\sisetup{detect-all}
\usepackage{chemfig}    %Write chemical formulas
\usepackage{mathtools}  %Basic math an extension of amsmath

% formatting support
\usepackage{multicol}   %Multiple cols body : command \multicols{#}{'text'}
\usepackage{multirow}   %Multiple row spanning in tables : \multirow{#}{width}{'text'}
\usepackage{enumitem}   %Enumeration control
\usepackage{amssymb}

\usepackage{hyperref} % hyperlinks
\usepackage{soul} % highlighting with \hl{command}


% font and date format
\usepackage{newtxtext}
\setkomafont{disposition}{\bfseries}
\usepackage{fancyref}           %Automatically adds Table (tab:' ') or Figure (fig:' ')
\usepackage{texdate}
\initcurrdate
\def\setdateformat{e\ b\ y\ }
\renewcommand{\headfont}{\normalfont}
\renewcommand{\footfont}{\normalfont}

% useful commands
\newcommand{\biu}[1]{\textbf{\emph{\underline{#1}}}}

\newcommand{\centerframe}[1]{ % this command makes a box around the content
    {\centering\fbox{\begin{minipage}{0.975\columnwidth}#1\end{minipage}}}}

\newcommand{\linespace}{5mm}

\newcommand{\blank}{\rule{60mm}{0.4pt}}

\newlist{checklist}{itemize}{1}
\setlist[checklist]{label=$\square$}


% document commands
\ihead{Name:}
\chead{Hour:}
\ohead{Date:\hspace{40mm}}
\ifoot{Grade: \hspace{40mm}Why:}
\cfoot{}
\ofoot{\thepage}
\newcommand{\maintitle}{E/M Assessment Due 12/13 Nov }

%===================================
\begin{document}
\section*{\maintitle}

\section{Engineering Design}
Describe why 3 reasons why designed succeeded/failed,
	describe 3 improvements,
	compare 3 points why a manufactured version is better,
	and explain why each of those 3 points are better.
\emph{See the proficiency scale for the expectations of 3, 2, 1}

\vspace{5mm}

{\centering
	\begin{tikzpicture}[scale=.6]
    \foreach \x in {0,...,32}
    \foreach \y in {0,...,28}
    {
  \fill (\x,\y) circle (0.03cm);
    }       
  \end{tikzpicture}

}
\clearpage
\newcommand{\numberoflines}{13}

\section{Electricity and Magnetism}
\subsection{Using a diagram}
Explain how your device works using arrows showing the direction of forces.

\vspace{2mm}

	\begin{tikzpicture}[scale=.6]
    \foreach \x in {0,...,32}
    \foreach \y in {0,...,\numberoflines}
    {
  \fill (\x,\y) circle (0.03cm);
    }       
  \end{tikzpicture}

\vfill
\subsection{In writing}
Explain how your device works in terms of electric current and magnetic fields.

\vspace{2mm}

	\begin{tikzpicture}[scale=.6]
    \foreach \x in {0,...,32}
    \foreach \y in {0,...,\numberoflines}
    {
  \fill (\x,\y) circle (0.03cm);
    }       
  \end{tikzpicture}
  














































\end{document}