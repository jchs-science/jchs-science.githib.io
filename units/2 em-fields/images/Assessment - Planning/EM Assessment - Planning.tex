\documentclass[14pt, fleqn, paper=letter, oneside]{scrartcl}

\newcommand{\includehead}{false}
\newcommand{\includefoot}{true}

% set basic page format
\usepackage[headsepline=\includehead, footsepline=\includefoot]{scrlayer-scrpage} 
\usepackage[margin=0.5in,
%    footskip=1.5\baselineskip, headsep=0.5\baselineskip,
    includehead=\includehead, includefoot=\includefoot]{geometry}   %Fixed margins
\usepackage[compact]{titlesec}

%\usepackage{setspace}
%\onehalfspace
%\doublespace    

% image support
\renewcommand{\topfraction}{0.85}   %Fixes float spacing
\renewcommand{\textfraction}{0.1}
\renewcommand{\floatpagefraction}{0.75}
\usepackage{graphicx}
	\graphicspath{{/Users/fred/Library/TeXShop/Images/}{./Images/}}
\usepackage[space]{grffile}		

% symbol support
\usepackage{siunitx}    %SI unit : \si{\'unit'} or \SI{#}{\'unit'}
\sisetup{detect-all}
\usepackage{chemfig}    %Write chemical formulas
\usepackage{mathtools}  %Basic math an extension of amsmath

% formatting support
\usepackage{multicol}   %Multiple cols body : command \multicols{#}{'text'}
\usepackage{multirow}   %Multiple row spanning in tables : \multirow{#}{width}{'text'}
\usepackage{enumitem}   %Enumeration control
\usepackage{amssymb}

\usepackage{hyperref} % hyperlinks
\usepackage{soul} % highlighting with \hl{command}


% font and date format
\usepackage{newtxtext}
\setkomafont{disposition}{\bfseries}
\usepackage{fancyref}           %Automatically adds Table (tab:' ') or Figure (fig:' ')
\usepackage{texdate}
\initcurrdate
\def\setdateformat{e\ b\ y\ }
\renewcommand{\headfont}{\normalfont}
\renewcommand{\footfont}{\normalfont}

% useful commands
\newcommand{\biu}[1]{\textbf{\emph{\underline{#1}}}}

\newcommand{\centerframe}[1]{ % this command makes a box around the content
    {\centering\fbox{\begin{minipage}{0.975\columnwidth}#1\end{minipage}}}}

\newcommand{\linespace}{5mm}

\newcommand{\blank}{\rule{60mm}{0.4pt}}

\newlist{checklist}{itemize}{1}
\setlist[checklist]{label=$\square$}


% document commands
%\ihead{Name:}
%\chead{Hour:}
%\ohead{Date:}
\ifoot{Revised \printdate}
\cfoot{\maintitle}
\ofoot{\thepage}
\newcommand{\maintitle}{E/M Assessment Due 12/13 Nov - Planning Guide}

%===================================
\begin{document}
\section{\maintitle}

\subsection{Question}
How can you build a functioning fan or speaker to help explain electric currents and magnetic fields?

\subsection{Prompt}
They are hosting a party for Sundown Salute in Junction City.
Your group has been tasked with creating speakers for the music OR fans to cool people off in the heat of the summer.
You must decide which job you are going to take and build the fan or speaker using the materials provided.
However, the city has given you a budget to design these projects and you only have \$10 to buy materials.
You must work together as a group to design this project within the allotted budget. 

\subsection{Basic Requirements}
\begin{description}
\item[Option 1 -] Research and design a working fan that will spin.
\item[Option 2 -] Research and design a speaker that will play the music.
\end{description}

\subsection{Materials List}
With your \$10 you can purchase:

\begin{table}[h]
\centering
\begin{tabular}{|l|l|l|l|}
\hline
\textbf{Item}          & \textbf{Cost} & \textbf{Item}   & \textbf{Cost} \\ \hline
Plastic cup            & \$1           & Styrofoam plate & \$1           \\ \hline
2 large paper clips    & \$1           & 4 post-it notes & \$1           \\ \hline
Rubber band            & \$1           & Water bottle    & \$2           \\ \hline
2 push pins            & \$1           & Nail            & \$1           \\ \hline
Donut magnet           & \$1           & 24-gauge wire   & \$1           \\ \hline
Neodymium magnet       & \$2           & 16-gauge wire   & \$2           \\ \hline
Small neodymium magnet & \$1           & Sharpie         & \$1           \\ \hline
Ferrite magnet         & \$1           & Use of pliers   & \$1           \\ \hline
Advice from adult      & \$1           &                 &               \\ \hline
\end{tabular}
\end{table}

\clearpage
\section{Proficiency Scale}

% Level4
\begin{minipage}[t]{0.48\textwidth}
\hfill \textbf{Engineering Design - 4}\hfill\mbox{}

\vspace{-4mm}
\begin{checklist}[leftmargin=*]
\item Describe 3 reasons why the fan/speaker passed/failed
\item Describe 3 improvements
\item Compare 3 points why a manufactured fan/speaker works better
\item Describe WHY for each of the 3 points

\end{checklist}

\end{minipage}
\hfill\vline\hfill
\begin{minipage}[t]{0.48\textwidth}
\hfill \textbf{Electricity and Magnetism - 4}\hfill\mbox{}

\vspace{-4mm}
\begin{checklist}[leftmargin=*]
\item In writing - explain how their fan/speaker works in terms of electric current and magnetic fields with no errors
\item Diagram - explain how their fan/speaker works; using arrows showing direction of forces

\end{checklist}

\end{minipage}

\vspace{2mm}
\rule{\textwidth}{1pt}


% Level 3
\noindent
\begin{minipage}[t]{0.48\textwidth}
\hfill \textbf{Engineering Design - 3}\hfill\mbox{}

\vspace{-4mm}
\begin{checklist}[leftmargin=*]
\item Describe 2 reasons why the fan/speaker passed/failed
\item Describe 2 improvements

\end{checklist}

\end{minipage}
\hfill\vline\hfill
\begin{minipage}[t]{0.48\textwidth}
\hfill \textbf{Electricity and Magnetism - 3}\hfill\mbox{}

One OR the other, with no errors

\vspace{-4mm}
\begin{checklist}[leftmargin=*]
\item In writing - explain how their fan/speaker works in terms of electric current and magnetic fields
\item Diagram - explain how their fan/speaker works; using arrows showing direction of forces

\end{checklist}

\end{minipage}

\vspace{2mm}
\rule{\textwidth}{1pt}

% Level2
\noindent
\begin{minipage}[t]{0.48\textwidth}
\hfill \textbf{Engineering Design - 2}\hfill\mbox{}

\vspace{-4mm}
\begin{checklist}[leftmargin=*]
\item Describe 1 reasons why the fan/speaker passed/failed
\item Describe 1 improvements

\end{checklist}

\end{minipage}
\hfill\vline\hfill
\begin{minipage}[t]{0.48\textwidth}
\hfill \textbf{Electricity and Magnetism - 2}\hfill\mbox{}

One OR the other, with minimal errors

\vspace{-4mm}
\begin{checklist}[leftmargin=*]
\item In writing - explain how their fan/speaker works in terms of electric current and magnetic fields
\item Diagram - explain how their fan/speaker works; using arrows showing direction of forces

\end{checklist}

\end{minipage}

\vspace{2mm}
\rule{\textwidth}{1pt}

% Level1
\noindent
\begin{minipage}[t]{0.48\textwidth}
\hfill \textbf{Engineering Design - 1}\hfill\mbox{}

\vspace{-4mm}
\begin{checklist}[leftmargin=*]
\item Only identify if the fan/speaker was a success or failure but no reasons given why.

\end{checklist}

\end{minipage}
\hfill\vline\hfill
\begin{minipage}[t]{0.48\textwidth}
\hfill \textbf{Electricity and Magnetism - 1}\hfill\mbox{}

\vspace{-4mm}
\begin{checklist}[leftmargin=*]
\item Not explain how their fan/speaker works in terms of electric current and magnetic fields.

\item Diagram with no labels or no description.

\end{checklist}

\end{minipage}

\clearpage
\section{Pre-Planning}
\begin{enumerate}[leftmargin=*]
\item Research both how to make a speaker, and a fan.
Summarize what you find below.

\begin{minipage}[t]{0.48\textwidth}
{\centering
Speaker

\vspace{2mm}

	\begin{tikzpicture}[scale=.6]
    \foreach \x in {0,...,14}
    \foreach \y in {0,...,15}
    {
  \fill (\x,\y) circle (0.03cm);
    }       
  \end{tikzpicture}

}
\end{minipage}
\hfill\vline\hfill
\begin{minipage}[t]{0.48\textwidth}
{\centering
Fan

\vspace{2mm}

	\begin{tikzpicture}[scale=.6]
    \foreach \x in {0,...,14}
    \foreach \y in {0,...,15}
    {
  \fill (\x,\y) circle (0.03cm);
    }       
  \end{tikzpicture}

}
 
\end{minipage}

\item Which do you choose? \blank

\item Why?

\vspace{2mm}
{\centering
	\begin{tikzpicture}[scale=.6]
    \foreach \x in {0,...,30}
    \foreach \y in {0,...,15}
    {
  \fill (\x,\y) circle (0.03cm);
    }       
  \end{tikzpicture}
  
}

\clearpage
\item Make a drawing of your design

\vspace{2mm}
{\centering
	\begin{tikzpicture}[scale=.6]
    \foreach \x in {0,...,30}
    \foreach \y in {0,...,35}
    {
  \fill (\x,\y) circle (0.03cm);
    }       
  \end{tikzpicture}
  
}

\item Label the currents, magnetic fields, and forces in your design.

\clearpage
\item Is there any way you can improve your design? 
(Think about the electro-magnet you made, and the magnets you played with. 
What made the forces stronger for either of them?
Think back to what we did in mechanics too.)

\vspace{2mm}
{\centering
	\begin{tikzpicture}[scale=.6]
    \foreach \x in {0,...,30}
    \foreach \y in {0,...,16}
    {
  \fill (\x,\y) circle (0.03cm);
    }       
  \end{tikzpicture}
  
}

\item Write a description for how your design works.

\vspace{2mm}
{\centering
	\begin{tikzpicture}[scale=.6]
    \foreach \x in {0,...,30}
    \foreach \y in {0,...,16}
    {
  \fill (\x,\y) circle (0.03cm);
    }       
  \end{tikzpicture}
  
}

\item What do you need to purchase?

\begin{table}[ht]
\centering
\renewcommand{\arraystretch}{2}

\begin{tabular}{|p{60mm}|c|c|}
\hline
\textbf{Material}
	&\textbf{Amount}
	&\textbf{Cost}
\\\hline&&
\\\hline&&
\\\hline&&
\\\hline&&
\\\hline&&
\\\hline&&
\\\hline&&
\\\hline&&
\\\hline&&
\\\hline
\end{tabular}
\end{table}

\item Total Cost: \blank	

\item If you have any extra money, how might you spend it?

\vspace{2mm}
{\centering
	\begin{tikzpicture}[scale=.6]
    \foreach \x in {0,...,30}
    \foreach \y in {0,...,12}
    {
  \fill (\x,\y) circle (0.03cm);
    }       
  \end{tikzpicture}
  
}

\clearpage
\item Make a list of the things you need to have in your writeup to get the grade you want.

\vspace{2mm}
{\centering
	\begin{tikzpicture}[scale=.6]
    \foreach \x in {0,...,30}
    \foreach \y in {0,...,17}
    {
  \fill (\x,\y) circle (0.03cm);
    }       
  \end{tikzpicture}
  
}

\item On this page and the next, make a draft of your write-up

\vspace{2mm}
{\centering
	\begin{tikzpicture}[scale=.6]
    \foreach \x in {0,...,30}
    \foreach \y in {0,...,17}
    {
  \fill (\x,\y) circle (0.03cm);
    }       
  \end{tikzpicture}
  
}

\clearpage
\vspace{2mm}
{\centering
	\begin{tikzpicture}[scale=.6]
    \foreach \x in {0,...,30}
    \foreach \y in {0,...,36}
    {
  \fill (\x,\y) circle (0.03cm);
    }       
  \end{tikzpicture}
  
}

\item Grade yourself to make sure you included everything you need.

\end{enumerate}













































\end{document}