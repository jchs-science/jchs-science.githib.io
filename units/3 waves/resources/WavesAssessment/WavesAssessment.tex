\documentclass[12pt, fleqn, paper=letter, oneside]{scrartcl}

\newcommand{\includehead}{false}
\newcommand{\includefoot}{true}

% set basic page format
\usepackage[headsepline=\includehead, footsepline=\includefoot]{scrlayer-scrpage} 
\usepackage[margin=0.5in,
%    footskip=1.5\baselineskip, headsep=0.5\baselineskip,
    includehead=\includehead, includefoot=\includefoot]{geometry}   %Fixed margins
\usepackage[compact]{titlesec}

\usepackage{setspace}
\onehalfspace
%\doublespace    

% image support
\renewcommand{\topfraction}{0.85}   %Fixes float spacing
\renewcommand{\textfraction}{0.1}
\renewcommand{\floatpagefraction}{0.75}
\usepackage{graphicx}
	\graphicspath{{/Users/fred/Library/TeXShop/Images/}{./Images/}}
\usepackage[space]{grffile}		

% symbol support
\usepackage{siunitx}    %SI unit : \si{\'unit'} or \SI{#}{\'unit'}
\sisetup{detect-all}
\usepackage{chemfig}    %Write chemical formulas
\usepackage{mathtools}  %Basic math an extension of amsmath

% formatting support
\usepackage{multicol}   %Multiple cols body : command \multicols{#}{'text'}
\usepackage{multirow}   %Multiple row spanning in tables : \multirow{#}{width}{'text'}
\usepackage{enumitem}   %Enumeration control
\usepackage{hyperref} % hyperlinks
\usepackage{soul} % highlighting with \hl{command}
\usepackage{amssymb}


% font and date format
\usepackage{newtxtext}
\setkomafont{disposition}{\bfseries}
\usepackage{fancyref}           %Automatically adds Table (tab:' ') or Figure (fig:' ')
\usepackage{texdate}
\initcurrdate
\def\setdateformat{e\ b\ y\ }
\renewcommand{\headfont}{\normalfont}
\renewcommand{\footfont}{\normalfont}

% useful commands
\newcommand{\biu}[1]{\textbf{\emph{\underline{#1}}}}
\newcommand{\centerframe}[1]{ % this command makes a box around the content
    {\centering\fbox{\begin{minipage}{0.975\columnwidth}#1\end{minipage}}}}
\newcommand{\linespace}{5mm}
\newcommand{\blank}{\rule{60mm}{0.4pt}}
\newlist{checklist}{itemize}{1}
\setlist[checklist]{label=$\square$}


% document commands
%\ihead{Name:}
%\chead{Hour:}
%\ohead{Date:}
\ifoot{Revised \printdate}
\cfoot{\maintitle}
\ofoot{\thepage}
\newcommand{\maintitle}{Waves Assessment - Worksheet}
\newcommand{\blanks}{\hrulefill

\rule{\textwidth}{0.4pt}
\rule{\textwidth}{0.4pt}
\rule{\textwidth}{0.4pt}
\rule{\textwidth}{0.4pt}
\rule{\textwidth}{0.4pt}
}

\newcommand{\evidencesource}[1]{
%\noindent\rule{\textwidth}{0.4pt}
\vline
\begin{minipage}[t]{0.48\textwidth}
\textbf{Evidence #1:} \hrulefill

\rule{\textwidth}{0.4pt}
\rule{\textwidth}{0.4pt}
\rule{\textwidth}{0.4pt}
\rule{\textwidth}{0.4pt}
\rule{\textwidth}{0.4pt}
\textbf{How it supports your choice:} \hrulefill

\rule{\textwidth}{0.4pt}
\rule{\textwidth}{0.4pt}
\rule{\textwidth}{0.4pt}
\rule{\textwidth}{0.4pt}
\rule{\textwidth}{0.4pt}
\end{minipage}
\hfill\vline\hfill
\begin{minipage}[t]{0.48\textwidth}
\textbf{Source:} \hrulefill

\rule{\textwidth}{0.4pt}
\rule{\textwidth}{0.4pt}
\textbf{Why is it a good source?} \hrulefill

\rule{\textwidth}{0.4pt}
\rule{\textwidth}{0.4pt}
\rule{\textwidth}{0.4pt}
\rule{\textwidth}{0.4pt}
\rule{\textwidth}{0.4pt}
\rule{\textwidth}{0.4pt}
\rule{\textwidth}{0.4pt}
\rule{\textwidth}{0.4pt}
\end{minipage}
}

%===================================
\begin{document}
\section*{\maintitle}
You want to start your own cellular network business and so you are going to purchase some wireless spectrum.
Your most expensive investment will be your frequency choice.
You can pick from any of the available options.

They are any place on the table labeled 'No':

\url{https://en.wikipedia.org/wiki/Cellular_frequencies_in_the_US}

\subsection*{Rubric and Guide}
For the full rubric go here: 
\url{https://jchs-science.github.io/waves/assessment}

The \hl{Wave Behaviors} will be a quiz that you do in class.
The \hl{Scientific Reporting} will be this worksheet.
If you want a 4, you will have to do more, see the rubric.

This worksheet is to help you collect \hl{EVIDENCE} and \hl{SOURCES} to support your choice of frequency.
You \textbf{MUST} explain how your evidence supports your choice, \textbf{AND} why your source is a valid source.
For help with both of those, go here:
\url{https://jchs-science.github.io/help}

\subsection*{Pick a frequency}
After you have done any pre-research you want to do, go here and pick a frequency that has a 'No' in it.
\url{https://en.wikipedia.org/wiki/Cellular_frequencies_in_the_US}

\textbf{Chosen Frequency:} \blank \si{\mega\hertz}

\subsection*{Evidence and Sources}

\vline
\begin{minipage}[t]{0.48\textwidth}
\textbf{Evidence \#1:} \hrulefill

\rule{\textwidth}{0.4pt}
\rule{\textwidth}{0.4pt}
\rule{\textwidth}{0.4pt}
\rule{\textwidth}{0.4pt}
\rule{\textwidth}{0.4pt}
\textbf{How it supports your choice:} \hrulefill

\rule{\textwidth}{0.4pt}
\rule{\textwidth}{0.4pt}
\rule{\textwidth}{0.4pt}
\rule{\textwidth}{0.4pt}
\rule{\textwidth}{0.4pt}
\end{minipage}
\hfill\vline\hfill
\begin{minipage}[t]{0.48\textwidth}
\textbf{Source:} \url{https://jchs-science.github.io/waves/assessment}

\textbf{Why is it a good source?}
Mr. Hicks said so.
(Note, this only works for this exact website.
EVEN if you pull other information from the class website you should find another source to verify it.)

\end{minipage}


\clearpage
\evidencesource{\#2}
\vfill
\evidencesource{\#3}
\vfill
\evidencesource{\#4}







































\end{document}