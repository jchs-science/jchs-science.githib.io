\documentclass[14pt, fleqn, paper=letter, oneside]{scrartcl}

\newcommand{\includehead}{true}
\newcommand{\includefoot}{true}

% set basic page format
\usepackage[headsepline=\includehead, footsepline=\includefoot]{scrlayer-scrpage} 
\usepackage[margin=0.5in,
%    footskip=1.5\baselineskip, headsep=0.5\baselineskip,
    includehead=\includehead, includefoot=\includefoot]{geometry}   %Fixed margins
\usepackage[compact]{titlesec}

%\usepackage{setspace}
%\onehalfspace
%\doublespace    

% image support
\renewcommand{\topfraction}{0.85}   %Fixes float spacing
\renewcommand{\textfraction}{0.1}
\renewcommand{\floatpagefraction}{0.75}
\usepackage{graphicx}
	\graphicspath{{/Users/fred/Library/TeXShop/Images/}{./Images/}}
\usepackage[space]{grffile}		

% symbol support
\usepackage{siunitx}    %SI unit : \si{\'unit'} or \SI{#}{\'unit'}
\sisetup{detect-all}
\usepackage{chemfig}    %Write chemical formulas
\usepackage{mathtools}  %Basic math an extension of amsmath

% formatting support
\usepackage{multicol}   %Multiple cols body : command \multicols{#}{'text'}
\usepackage{multirow}   %Multiple row spanning in tables : \multirow{#}{width}{'text'}
\usepackage{enumitem}   %Enumeration control
\usepackage{hyperref} % hyperlinks
\usepackage{soul} % highlighting with \hl{command}


% font and date format
\usepackage{newtxtext}
\setkomafont{disposition}{\bfseries}
\usepackage{fancyref}           %Automatically adds Table (tab:' ') or Figure (fig:' ')
\usepackage{texdate}
\initcurrdate
\def\setdateformat{e\ b\ y\ }
\renewcommand{\headfont}{\normalfont}
\renewcommand{\footfont}{\normalfont}

% useful commands
\newcommand{\biu}[1]{\textbf{\emph{\underline{#1}}}}
\newcommand{\centerframe}[1]{ % this command makes a box around the content
    {\centering\fbox{\begin{minipage}{0.975\columnwidth}#1\end{minipage}}}}

% document commands
\ihead{CLASSROOM COPY}
\chead{Record your observations in your lab notebook}
\ohead{CLASSROOM COPY}
\ifoot{Revised \printdate}
\cfoot{\maintitle}
\ofoot{\thepage}
\newcommand{\maintitle}{The Author}

\setlist[enumerate,1]{label = \textbf{Q\arabic*} -,
                      ref   = \arabic*}

%===================================
\begin{document}
\section*{Wave Properties Lab}
The purpose of this lab is to familiarize yourself with the different words used to describe waves.

Follow the directions and answer the questions in your lab notebook.

\section*{Vocabulary}
\begin{description}
\item[Frequency -] How many times a waves passes a given point in a second.

\item[Wavelength -] The distance from one peak to another peak.

\item[Amplitude -] The 'height' of the wave, how much thicker it is than a flat line.

\end{description}


\section{Rope Demonstration}
Your teacher and a student volunteer will demonstrate a simple wave with a rope.

\hl{Answer the questions below in your lab notebook.}
\begin{enumerate}
\item Sketch the wave.
\item Label which direction the wave moves.
\item Where does the energy come from?
\end{enumerate}

\section{Wave Properties - Yarn}

\begin{enumerate}[resume]
\item 
\end{enumerate}











































\end{document}